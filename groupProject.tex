\documentclass{article}


%\usepackage[margin=0.75in]{geometry}
\usepackage{amsmath}
\usepackage{amssymb}
\renewcommand{\vec}[1]{\mathbf{#1}}
\usepackage{multicol}
\let\oldhat\hat
\newcommand{\basis}[1]{\oldhat{\mathbf{#1}}}
%\setlength{\columnsep}{1cm}
\usepackage{times}
\usepackage{graphicx}
\usepackage{tikz}
\usepackage{subcaption}
\usepackage{listings}
\setlength{\parindent}{0pt}



\begin{document}

\title{An Assessment of Numerical Methods in Electrostatics}
\author{S. Gallacher, A. Gosnay, A. Hameed, M. Nairn}
\maketitle


\tableofcontents

\newpage

\begin{abstract}



\end{abstract}




\section{Introduction}

DISCUSS NUMERICAL METHODS USES IN GENERAL AND HOW THEY BECOME USEFUL IN ELECTROSTATICS


\section{Maxwell's Equations in Electrostatics}

In electrostatics for free space, Maxwell's equations concerning the electric field, $\vec{E}$, take the form,

$$
\nabla \cdot \vec{E} = 0 \quad \quad \nabla \times \vec{E} = \vec{0}.
$$

Since the curl of $\vec{E}$ vanishes, the field may be written in terms of some potential, $V$, with,

\begin{equation}
\vec{E} = - \nabla V = -\frac{\partial V}{\partial x} \basis{e}_x - \frac{\partial V}{\partial y} \basis{e}_y \quad \text{(2D)}.
\label{eFieldPotential}
\end{equation}

$V$ is known as the electrostatic potential. Futhermore, taking the divergence of Eqn.~\ref{eFieldPotential} and using the above Maxwell relation, $V$ may be written as,

\begin{equation} 
\nabla^2 V = \frac{\partial^2 V}{\partial x^2} + \frac{\partial^2 V}{\partial y^2} = 0 \quad \text{(2D)},
\label{laplace}
\end{equation}

\noindent which is Laplace's equation. Hence, solving Eqn.~\ref{laplace} with specified boundary conditions for a given system will allow $V$ to be determined which will in turn allow $\vec{E}$ for the system to be determined.

\section{Finite Difference Methods}

A finite difference method is a approximation that allows derivatives of a function to be computed numerically. The analytical derivative of a function $f$ is defined as the limit,

$$
f'(x) = \lim_{h \to 0} \frac{f(x+h)-f(x)}{h}.
$$

A finite difference method approximates the derivative by assuming the difference $h$ is not infinitesimal but is a small finite value giving,

$$
f'(x) \approx \frac{f(x+h)-f(x)}{h},
$$

with the approximation becoming better the smaller the value of $h$. The result above may also be examined by considering the Taylor expansion,

$$
f(x+h) = f(x) + hf'(x) + \frac{h^2}{2}f''(x) + \frac{h^3}{3}f'''(x) + \ldots 
$$


\subsection{Solving Laplace's Equation}



\subsection{Obtaining the Electric Field from the Electrostatic Potential}

With a known electrostatic potential, the electric field can be obtained numerically from Eqn.~\ref{eFieldPotential} but requires a method of computing partial derivatives of $V$. This can be done, again, by employing finite difference methods.



\section{Implementation}

\subsection{Setting Initial Conditions}




\subsection{Finding the Electrostatic Potential - Gauss-Seidel}


\subsection{Finding the Electrostatic Potential - Jacobian}



\subsection{Finding the Electric Field}



\subsection{Output}



\section{Co-Axial Cylinders}

DISCUSS THE GEOMETRY

INCLUDE SCOTTS INITIAL CONDITIONS IMAGE



\subsection{Analytical Solutions}


\subsection{Numerical Solutions}



\subsection{Evaluation}




\section{Perturbed Parallel Plates}

DISCUSS THE GEOMETRY

INCLUDE SCOTTS INITIAL CONDITIONS IMAGE



\subsection{Analytical Solutions}


\subsection{Numerical Solutions}



\subsection{Evaluation}



\section{Gas Electron Multiplier}

DISCUSS THE GEOMETRY AND APPLICATIONS OF ELECTRON MULTIPLIER

INCLUDE SCOTTS INITIAL CONDITIONS IMAGE




\subsection{Numerical Solutions}



\subsection{Evaluation}



\section{Conclusions and Discussion}






\end{document}
