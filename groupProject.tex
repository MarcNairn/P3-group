\documentclass{article}


\usepackage[margin=0.75in]{geometry}
\usepackage{amsmath}
\usepackage{amssymb}
\renewcommand{\vec}[1]{\mathbf{#1}}
\usepackage{multicol}
\let\oldhat\hat
\renewcommand{\hat}[1]{\oldhat{\mathbf{#1}}}
\setlength{\columnsep}{1cm}
\usepackage{times}
\usepackage{graphicx}
\usepackage{tikz}
\usepackage{subcaption}
\usepackage{listings}
\setlength{\parindent}{0pt}



\begin{document}


\title{An Assessment of Numerical Methods in Electrostatics}
\author{S. Gallacher, A. Gosnay, A. Hameed, M. Nairn}
\maketitle


\begin{abstract}



\end{abstract}


\begin{multicols*}{2}

\section*{Introduction}




\section*{Maxwell's Equations in Electrostatics}

In electrostatics for free space, Maxwell's equations concerning the electric field, $\vec{E}$, take the form,

$$
\nabla \cdot \vec{E} = 0 \quad \quad \nabla \times \vec{E} = 0.
$$

Since the curl of $\vec{E}$ vanishes, the field may be written in terms of some potential, $V$, with,

\begin{equation}
\vec{E} = - \nabla V.
\label{eFieldPotential}
\end{equation}

$V$ is known as the electrostatic potential. Futhermore, taking the divergence of Eqn.~\ref{eFieldPotential} and using the above Maxwell relation, $V$ may be written as,

\begin{equation} 
\nabla^2 V = 0,
\label{laplace}
\end{equation}

\noindent which is Laplace's equation. Hence, solving Eqn.~\ref{laplace} with specified boundary conditions for a given system will allow $V$ to be determined which will in turn allow $\vec{E}$ for the system to be determined by Eqn~\ref{eFieldPotential}.

\section*{Finite Difference Methods}




\end{multicols*}


\end{document}
